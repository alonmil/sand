\documentclass[12pt]{article}
\textwidth = 6.5 in
\textheight = 9 in
\oddsidemargin = 0.0 in
\evensidemargin = 0.0 in
\topmargin = 0.15 in
\headheight = 0.0 in
\headsep = 0.0 in
\parskip = 0.2in
\parindent = 0.0in

\usepackage{amssymb}
\usepackage{amsmath}
\usepackage{enumerate}
\usepackage{amsthm}
\usepackage{algorithm, algpseudocode}


\newenvironment{solution}
	{\begin{center}
	\begin{tabular}{|p{0.9\textwidth}|}
	\hline\\
	\textbf{Solution:}
	}
	{\\\\
	\hline
	\end{tabular}
	\end{center}
	}

\begin{document}
{\huge CSE 332 Summer 18\\
Section 03}
\section{Solving Recurrences}
For each of the following recurrences, use the tree method to find a closed form of the recurrence. 

When using the tree method, you should do the following steps.
\begin{enumerate}
	\setcounter{enumi}{-1}
	\item Draw at least the first two levels of the recursion tree, and the leaf level of the tree. 
	\item Let the root node be at level $0$. Give a formula for the size of the input at level $i$. 
	\item What is the number of nodes at level $i$? 
	\item What is the work done at the $i^{\text{th}}$ recursive level? 
	\item What is the last level of the tree? 
	\item What is the work done at the base case? 
	\item Write an expression for the total work done. Your expression should include a summation. 
	\item Find a ``closed form'' of the formula in the previous part. To qualify as a closed form, it must not have any summations or recursion, but it does not have to ``look nice.'' 
	\item If possible, use Master Theorem to sanity check your answer.
\end{enumerate}

\newpage
\begin{enumerate}[a)]
	\item $ T(n) = \begin{cases} 1 &\text{ if } n=1\\
				T(n/2) + 3 &\text{ otherwise}
				\end{cases}$
			\begin{solution}
				\begin{enumerate}[1.]
					
					\item $\frac{n}{2^i}$ 
					\item $1$
					\item $1 \cdot 3$					
					\item We want the $i$ such that $\frac{n}{2^i} = 1$. Solving we get $\log_2(n)$ as the last level.
                    \item $1 \cdot 1$
					\item $\sum\limits_{i=0}^{\log_2(n) - 1} 3 + 1$
					\item $3\log_2(n) + 1$
					\item Since $\log_2(1) = 0$ so the Master theorem says we should get $\Theta(n^0 \log n)$ i.e. $\Theta(\log n)$, which matches our answer from the last part.
				\end{enumerate}
			\end{solution}
	\item $ T(n) = \begin{cases} 1 &\text{ if } n=0\\
				T(n-1) + 2 &\text{ otherwise}
				\end{cases}$
				\begin{solution}
					\begin{enumerate}[1.]
						\item $n-i$
						\item $1$
						\item $2\cdot 1$
						\item We want the level $i$ at which $n-i = 0$, so we want level $n$.
						\item $1 \cdot 1$
						\item $\sum\limits_{i=0}^{n-1} 2 + 1$
						\item $2n+1$
						\item Master Theorem does not apply here :/
						
					\end{enumerate}
				\end{solution}
				\newpage
	\item $ T(n) = \begin{cases} 1 &\text{ if }n=1\\
					3T(n/3) + n &\text{ otherwise}
					\end{cases}$
				\begin{solution}
					\begin{enumerate}[1.]
						\item $\frac{n}{3^i}$
						\item $3^i$
						\item $\frac{n}{3^i} \cdot 3^i = n$
						\item The level $i$ at which $\frac{n}{3^i} = 1$ i.e. $\log_3(n)$. 
						\item $3^{\log_3(n)} \cdot 1 = n$
						\item $\sum\limits_{i=0}^{\log_3(n) - 1} n + n$ 
						\item $n\log_3(n) + n$
						\item $\log_3(3) = 1$, so Master Theorem says $\Theta(n \log n)$, which is consistent with our answer. 
					\end{enumerate}
				\end{solution}
				\newpage
	\item  $ T(n) = \begin{cases} 1 &\text{ if }n=3\\
					2T(n/3) + n &\text{ otherwise}
					\end{cases}$
					\begin{solution}
					\begin{enumerate}[1.]
						\item $\frac{n}{3^i}$
						\item $2^i$
						\item $2^i \frac{n}{3^i} = n\left(\frac{2}{3}\right)^i$
						\item The level $i$ such that $\frac{n}{3^i} = 3$ so $\log_3(n) - 1$ is the last level.
						\item $2^{\log_3(n) - 1} \cdot 1$
						\item $\sum\limits_{i=0}^{\log_3(n) - 2} n\left(\frac{2}{3}\right)^i + 2^{\log_3(n) - 1}$
						\item 
						Applying the finite geometric series formula we get:
						\[ n \frac{\left(\frac{2}{3}\right)^{\log_3(n) - 1 } - 1}{\frac{2}{3} - 1} +  2^{\log_3(n) - 1}\]
						Looks pretty ugly, but it's a closed form. So we'll stop here.
						\item $\log_3(2) < 1$ so we should get $\Theta(n)$. The crazy formula from the last part is actually $\Theta(n)$. The second term is on the order of $n^{\log_3(2)}$, which is asymptotically less than $n$ (since $\log_3(2) < 1$) In the first term, the denominator is negative, so it's really $n\left(c - c'\left(\frac{2}{3}\right)^{\log_3(n) - 1 }\right)$ (where $c,c'$ are constants. As $n$ gets larger, $\left(\frac{2}{3}\right)^{\log_3(n) - 1 }$ is getting smaller. So $cn$ really is the dominating term.
					\end{enumerate}
					\end{solution}
	\newpage
	\item $ T(n) = \begin{cases} 2 &\text{ if }n=4\\
					4T(n/2) + n^2 &\text{ otherwise}
					\end{cases}$
			\begin{solution}
				\begin{enumerate}[1.]
					\item $\frac{n}{2^i}$
					\item $4^i$
					\item $4^i \cdot \left(\frac{n}{2^i}\right)^2 = n^2$. 
					\item We want the level $i$ where $4 = \frac{n}{2^i}$, so we want level $\log_2(n) - 2$. 
					\item $4^{\log_2(n) - 2} \cdot 2 = 2^{2\log_2(n) - 3} = \frac{n^2}{8}$
					\item $\sum\limits_{i=0}^{\log_2(n) - 3}  n^2 + \frac{n^2}{8} $
					\item $ (\log_2(n) - 2)n^2 + \frac{n^2}{8}$
					\item $\log_2(4) = 2$, so Master Theorem predicts $\Theta(n^2 \log n)$, which matches our answer.
				\end{enumerate}
			\end{solution}
\end{enumerate}

\section{Writing Recurrences}
Answer the following questions about these pseudocode snippets.
In cases where you are describing the running time, you should describe the non-recursive work using a simple function. For example, if the total number of non-recursive operations was $2n+4$ you can describe this as just $c\cdot n$ (where $c$ is a constant, and we ignore the lower-order term). 

\begin{enumerate}[a)]
\item \begin{algorithm}
\begin{algorithmic}
	\Function{\texttt{F}}{$n$}
		\If{$n == 0$}
			\State \Return 1
		\Else
			\State \Return $2 \cdot \texttt{F}(n-1) + 1$
		\EndIf
	\EndFunction
\end{algorithmic}
\end{algorithm}

\begin{itemize}
	\item Write a recurrence to describe the output of the function.
	\begin{solution}
		\[ P(n) = \begin{cases} 1 &\text{ if } n=0 \\
						2P(n-1) + 1 &\text{ otherwise}
		\end{cases}\]
		
	
	\end{solution}
	
	\item Write a recurrence to describe the running time of the function.
	\begin{solution}
			\[ T(n) = \begin{cases} c_1 &\text{ if } n=0\\
								T(n-1) + c_2 &\text{ otherwise}
		\end{cases}\]
		where $c_1, c_2$ are constants.
	\end{solution}
\end{itemize}

\item \begin{algorithm}
\begin{algorithmic}
	\Function{\texttt{F}}{$n$}
		\If{$n == 0$}
			\State \Return 0
		\EndIf
		
		\State \texttt{result} $\gets 0$
		\For{$i$ from $0$ to $n-1$}
			\For{$j$ from $0$ to $i$}
				\State \texttt{result} $\gets$ \texttt{result} $+ j$
			\EndFor
		\EndFor
		\State \Return \texttt{F}($n/2$) + \texttt{result} + \texttt{F}($n/2)$
		\EndFunction
\end{algorithmic}
\end{algorithm}
\begin{itemize}
	\item Write a recurrence to model the running time of this function.
	\begin{solution}
		\[ T(n) = \begin{cases} c_1 &\text{ if }n=0\\
								2T(n/2) + c_2n^2 &\text{ otherwise}
								\end{cases}\]
		Where $c_1, c_2$ are constants.
	\end{solution}
	\item Find a Big-$\Theta$ bound on the running time. 
	\begin{solution}
		Applying the master theorem, since $\log_2(2) < 2$ is $\Theta(n^2)$. We could also solve the recurrence with the tree method to get this bound.
	\end{solution}
\end{itemize}
\newpage
\item \begin{algorithm}
\begin{algorithmic}
	\Function{\texttt{G}}{$n$}
		\If{$n \leq 1$}
			\State \Return 1000
		\EndIf
		\If{\texttt{G}($n/3$) $> 5$}
			\For{$i$ from $0$ to $n-1$}
				\State print "YAY!"
			\EndFor
			\State \Return $5 \cdot \texttt{G}(n/3)$
		\Else
			\For{$i$ from $0$ to $n^2-1$}
				\State print "YAY!"
			\EndFor
			\State \Return $4\cdot \texttt{G}(n/3)$ 
		\EndIf
		\EndFunction
\end{algorithmic}
\end{algorithm}
\begin{itemize}
	\item For what values of $n$ do we reach the ``else'' branch?
	\begin{solution}
		No values of $n$ use the else branch. It was a trick -- the else branch is actually dead code.
	\end{solution}
	\item Write a recurrence to describe the worst case running time of $\texttt{G}$. 
	\begin{solution}
		\[ T(n) = \begin{cases}
				1 &\text{ if } n \leq 1\\
				2T(n/3) + cn &\text{ otherwise}
		\end{cases}\]
		Where $c$ is some constant. 
	\end{solution}
	\item Find a big-$\Theta$ bound on the running time by solving the recurrence.
	\begin{solution}
		We actually did almost exactly this in Section 1 (part d), with just $n$ instead of $cn$, and hitting the base case a little earlier. The small difference in the definition doesn't affect the $\Theta(n)$ bound we got there.
	\end{solution}
\end{itemize}
\end{enumerate}
\end{document}