\documentclass[12pt]{article}
\textwidth = 6.5 in
\textheight = 9 in
\oddsidemargin = 0.0 in
\evensidemargin = 0.0 in
\topmargin = 0.15 in
\headheight = 0.0 in
\headsep = 0.0 in
\parskip = 0.2in
\parindent = 0.0in

\usepackage{amssymb}
\usepackage{amsmath}
\usepackage{enumerate}
\usepackage{amsthm}
\usepackage{comment}
\usepackage{color}
\definecolor{purple}{RGB}{51,0,111}

\newenvironment{solution}
	{\begin{center}
	\begin{tabular}{|p{0.9\textwidth}|}
	\hline\\
	\textbf{Solution:}
	}
	{\\\\
	\hline
	\end{tabular}
	\end{center}
	}

\begin{document}
\vspace*{-2em}
{\large CSE 332 Summer 18\\
Section 02}
\vspace{-1em}
	
\begin{center}
	{\huge \color{purple} $\mathcal{O}, \Omega,$ and $\Theta$ oh my!}
\end{center}

Give a formal proof of each of the following statements, along with the scratch-work to find the $c$ and $n_0$. 

\begin{enumerate}[(a)]
	\item $5n + 7$ is $O(n)$
		\begin{solution}
			Scratch work:\\
			$5n \leq 5\cdot n$ for all $n$\\
			$7 \leq n$ for $n \geq 7$\\
			So $5n + 7 \leq 6n$ for $n\geq 7$.
			
			\begin{proof}
				We take $c = 6$ and $n_0 = 7$. For $n \geq n_0$ we have both of the following inequalities:
				\[5n \leq 5n \text{ and }7 \leq n\] 
				Adding together the two inequailities we have:
				$5n + 7 \leq 6n$ as long as $n \geq n_0$, which is what we needed to show.
			\end{proof}
			
		\end{solution}
	\item $3n^2 - 17n$ is $O(n^2)$
		\begin{solution}
			Scratch work:\\
			$3n^2 \leq 3\cdot n^2$ for all $n$\\
			$-17n \leq 0\cdot n^2$ if $n \geq 1$
			Take $c = 3+0 = 3$ and $n_0 = 1$.
			
			\begin{proof}
				We take $c=3$ and $n_0=1$. 
				For $n$ at least $1$, $-17n$ is negative, so it is certainly at most $0 = 0n^2$, and $3n^2$ is always at most $3n^2$. Adding together these inequalities we get $3n^2 - 17n \leq 3n^2$ for $n \geq 1$, which is what we wanted to show.
			\end{proof}
		\end{solution}
		
	\item $\log_5(n)$ is $\Omega(\log_3(n))$
		\begin{solution}
			Scratch work:\\
			Applying the change of base formula, $\log_5(n) = \frac{\log_3(n)}{\log_3(5)}$.
			
			\begin{proof}
				We take $c = \frac{1}{\log_3(5)}$, and $n_0 = 1$. Applying the change-of-base formula:
				\[ \log_5(n) = \frac{\log_3(n)}{\log_3(5)} \geq c\cdot \log_3(n)\] for all $n\geq 1$.
			\end{proof}
		\end{solution}
	\item $2n^3 + 3$ is $\Theta(n^3)$
		\begin{solution}
			This is basically two proofs in one.
			
			Scratch work for $O$:\\
			$2n^3 \leq 2n^3$\\
			$3 \leq 3n^3$ for $n \geq 1$. 
			
			Scratch work for $\Omega$:\\
			$2n^3 \geq 2n^3$\\
			$3 \geq 0n^3$ 
			
			\begin{proof}
				To show $2n^3 + 3$ is $O(n^3)$, we take $c = 5$ and $n_0 = 1$.
				We have the following inequalities for $n\geq 1$:
				\[ 2n^3 \leq 2n^3 \text{ and } 3 \leq 3n^3 \]
				Adding these inequalities together gives: $2n^3 + 3 \leq 5n^3$, as required. Thus $2n^3 + 3$ is $O(n^3)$.
				
				To show $2n^3 + 3$ is $\Omega(n^3)$, we take $c = 2$ and $n_0 = 1$. 
				We have $2n^3 + 3 \geq 2n^3 = c\cdot n^3$, which is what we needed to show to conclude $2n^3 + 3$ is $\Omega(n^3)$.
				
				Combining these two statements we have $2n^3 + 3$ is $\Theta(n^3)$. 
			\end{proof}
		\end{solution}	
	
\end{enumerate}

\end{document}