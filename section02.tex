\documentclass[12pt]{article}
\textwidth = 6.5 in
\textheight = 9 in
\oddsidemargin = 0.0 in
\evensidemargin = 0.0 in
\topmargin = 0.15 in
\headheight = 0.0 in
\headsep = 0.0 in
\parskip = 0.2in
\parindent = 0.0in

\usepackage{amssymb}
\usepackage{amsmath}
\usepackage{enumerate}
\usepackage{amsthm}
\usepackage{comment}
\usepackage{color}
\definecolor{purple}{RGB}{51,0,111}
\usepackage{comment}

\newenvironment{solution}
	{\begin{center}
	\begin{tabular}{|p{0.9\textwidth}|}
	\hline\\
	\textbf{Solution:}
	}
	{\\\\
	\hline
	\end{tabular}
	\end{center}
	}

\begin{document}
\vspace*{-2em}
{\large CSE 332 Summer 18\\
Section 02}
\vspace{-1em}
	
\begin{center}
	{\huge \color{purple} $\mathcal{O}, \Omega,$ and $\Theta$ oh my!}
\end{center}

Give a formal proof of each of the following statements, along with the scratch-work to find the $c$ and $n_0$. 

\begin{enumerate}[(a)]
	\item $5n+7$ is $O(n)$
		\vspace{1.1in}
	\item $3n^2 - 17n$ is $O(n^2)$
		\vspace{1.1in}
		
	\item $\log_5(n)$ is $\Omega(\log_3(n))$
\vspace{1.1in}
	\item $2n^3 + 3$ is $\Theta(n^3)$
	
\end{enumerate}

\end{document}